% !TeX program = PdfLaTeX
% !TeX root = ../../Elaborati_Aerodinamica_Bruno_Spoti.tex

\chapter{La teoria globale}
In questo capitolo saranno esposti alcuni risultati fondamentali dell'aerodinamica, tratti dalla teoria globale che si basa sul principio secondo cui il fenomeno del volo può essere visto come il moto in un condotto a sezione costante in regime unidimensionale stazionario. Nel moto all'interno del tubo di flusso, gli effetti della viscosità sono trascurati, pertanto la resistenza che si considera in questo capitolo è solo di tipo indotto.\\ In primo luogo sarà ricavato il $C_{L_{\mathrm{max}}}$ del velivolo in condizioni di atterraggio a livello del mare, successivamente saranno mostrate le variazioni di alcune grandezze fondamentali, quali $V_{\mathrm min}$, $C_L$, $C_{D_i}$, $D_i$,$\beta$, $\varDelta v$, con quota, peso e velocità. 