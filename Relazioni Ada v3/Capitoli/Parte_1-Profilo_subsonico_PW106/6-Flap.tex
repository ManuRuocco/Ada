% !TeX program = PdfLaTeX
% !TeX root = ../../Elaborati_Aerodinamica_Bruno_Spoti.tex
\chapter{Modifiche della geometria}

In questo capitolo verranno apportate delle modifiche alla geometria del profilo tramite XFOIL al fine di studiarne la configurazione con {\itshape flap} e Alettoni.\\ Tramite la funzione {\itshape f}, nel menu {\itshape GDES} di XFOIL è possibile deflettere parte del profilo così da simulare un {\itshape plain flap}.
Di seguito saranno riportate le curve di portanza, le polari, la distribuzione del coefficiente di pressione e i principali risultati numerici per diverse deflessioni dignificative delle superfici mobili.

\section {Configurazione con {\itshape Flap}}

Per studiare il comportamento del profilo con {\itshape flap} è stata prevista una deflessione di  $25^\circ $, compatibilmente con dati verosimili in condizione di atterraggio. La corda del flap è stata scelta pari al 30\% della corda del profilo.\\
Lo scopo del flap è quello di aumentare il $C_{l_{\mathrm{max}}}$ del profilo, ottenendo, però, anche uno stallo ad angoli d'attacco più bassi. Con flap parzialmente aperti (dai $10^\circ $ ai $20^\circ $) l’effetto risultante è un forte aumento di portanza e un relativamente piccolo incremento di resistenza. Per flap con deflessioni più ampie si ha anche un forte aumento di resistenza. In tabella~\ref{tab:flap} sono riportati gli effetti della deflessione dei flap, in termini di $C_{l_{max}}$ e angolo di stallo per $Re=1\times10^6$, in caso incomprimibile.\\

\begin{table} [!h]\centering \rowcolors{1}{}{grigio_chiaro}
\begin{tabular}{c S c c}
\toprule
\emph{Angolo di deflessione dei flap} &  $C_{l_{\mathrm{max}}}$  &  ${\alpha}_{\mathrm{stall}}$ & ${\alpha}_{\mathrm{zl}}$  \\ 
\midrule
$\delta_{\mathrm{flap}}=0^\circ$ & 1.37 & $13.1^\circ$  & $-0.873^\circ$  \\
$\delta_{\mathrm{flap}}=25^\circ$ & 1.75 & $7.41^\circ$ & $-14.7^\circ$  \\
\bottomrule
\end{tabular}
\caption {Profilo alare PW106, $C_{l_{\mathrm{max}}}$, ${\alpha}_{\mathrm{stall}}$ e ${\alpha}_{\mathrm{zl}}$ per flap retratto e deflesso, $ Re=1\times10^6$.}
\label{tab:flap}
\end{table}


\begin{figure}
\centering
\begin{tikzpicture} 
\begin{axis} [ 
ylabel style={rotate=-90}, xmin=0, 
xmax=1, 
ymin=-0.2,
 ymax=0.1,
 xlabel=$ \frac {x}{c}$, 
ylabel=$ \frac {z}{c}$,
ytick={-0.2,-0.1,0,0.1},
yticklabels={$-0.2$,$-0.1$,$0$,$0.1$},
width=13cm,
 height=3.9 cm,
scale only axis,
grid=major] 
\addplot [black,solid,very thin]
file{images/fileDat/FlapsAlettoni/ProfiloPW106Chiuso.dat};
\addplot [black,solid,very thick]
file{images/fileDat/FlapsAlettoni/PW106FlapTakeOff.dat};
\end{axis}
\end{tikzpicture}
\caption{\footnotesize Profilo alare PW106, configurazione con flap per $\delta_{flap}=25 ^\circ$  }
\end{figure}


\begin{figure} [H]
\centering
\begin{tikzpicture} 
\begin{axis} [ 
legend style={at={(0.35,0.98)}},
ylabel style={rotate=-90}, xmin=-18, 
xmax=18, 
ymin=0,
ymax=2,
xlabel=${\alpha}$,
ylabel=$C_l$ ,
width=10cm,
height=12.5 cm,
scale only axis,
grid=major] 
\addplot [black, smooth, thick]
file{images/fileDat/FlapsAlettoni/clFlapClean.dat};
\addplot [black,smooth,mark=*, thick]
file{images/fileDat/FlapsAlettoni/ClFlap25.dat};
\legend {$\delta_{flap}=0 ^\circ$,$\delta_{flap}=25 ^\circ$}
\end{axis}
\end{tikzpicture} 
\caption{\footnotesize Profilo alare PW106,  confronto delle curve di portanza del profilo per $ Re=1\times10^6$ per flap deflessi e in configurazione pulita. XFOIL 6.99 }
\label{fig:flappi}
\end{figure}

\begin{figure} [H]
\centering
\begin{tikzpicture} 
\begin{axis} [ 
ylabel style={rotate=-90}, xmin=0, 
xmax=1, 
ymin=-7,
ymax=1,
xlabel=$\frac{x}{c}$, 
ylabel=$C_p$ ,
 y dir=reverse,
width=12cm,
height=8 cm,
scale only axis,
grid=major] 
\addplot [black,thin]
file{images/fileDat/FlapsAlettoni/Cp_alfa_4_Re_1_10e6dorso.dat};
\addplot [black,ultra thick]
file{images/fileDat/FlapsAlettoni/PW106CP4degdorso.dat};
\addplot [black,thin,dashed]
file{images/fileDat/FlapsAlettoni/Cp_alfa_4_Re_1_10e6ventre.dat};
\addplot [black,ultra thick,dashed]
file{images/fileDat/FlapsAlettoni/PW106CP4degventre.dat};
\legend {$\delta_{flap}=0 ^\circ$,  $\delta_{flap}=25 ^\circ$}
\end{axis}
\end{tikzpicture}
\caption{\footnotesize Profilo alare PW106, confronto del Coefficiente di Pressione con flap deflessi $\delta_{flap}=25 ^\circ$ e configurazione pulita. $Re=1\times10^6$ $\alpha =4 ^\circ$ }
\end{figure}

\begin{figure} [H]
\centering
\begin{tikzpicture} 
\begin{axis} [ 
legend style={at={(0.98,0.20)}},
ylabel style={rotate=-90}, xmin=0, 
ymin=-1,
ymax=2,
xlabel=$C_d$,
ylabel=$C_l$ ,
width=11cm,
height=8cm,
scale only axis,
grid=major] 
\addplot [black, smooth,thick]
file{images/fileDat/FlapsAlettoni/cd_cl_10_6.dat};
\addplot [black,smooth,mark=square]
file{images/fileDat/FlapsAlettoni/CdFlapTo.dat};
\legend {$\delta_{flap}=0 ^\circ$,$\delta_{flap}=25 ^\circ$}
\end{axis}
\end{tikzpicture}
\caption{\footnotesize  Profilo alare PW106, confronto delle polari per $ Re=1\times10^6$ con flap deflessi e in configurazione pulita }
\end{figure}

\begin{figure} [H]
\centering
\begin{tikzpicture} 
\begin{axis} [ 
ylabel style={rotate=-90}, xmin=0, 
xmax=35, 
ymin=-20,
ymax=0,
xlabel=$\delta_{flap}$,
ylabel= ${\alpha}_{\mathrm{0l}}$ ,
width=11cm,
height=6cm,
scale only axis,
grid=major] 
\addplot [black, smooth,thick]
file{images/fileDat/FlapsAlettoni/VariazioneZeroLiftDeltaFlap.dat};
\end{axis}
\end{tikzpicture}
\caption{\footnotesize Profilo alare PW106, variazione del  ${\alpha}_{\mathrm{0l}}$ con l 'angolo di deflessione del flap per $ Re=1\times10^6$}
\end{figure}

%\begin{figure} [H]
%\centering
%\begin{tikzpicture} 
%\begin{axis} [ 
%ylabel style={rotate=-90}, xmin=0, 
%xmax=50, 
%ymin=1.5,
%ymax=2.1,
%ytick={1.5,1.6,1.7,1.8,1.9,2,2.1},
%xlabel=$\delta_{flap}$,
%ylabel=$C_{l_{max}}$ ,
%width=11cm,
%height=6cm,
%scale only axis,
%grid=major] 
%\addplot [black, smooth,thick]
%file{immagini/clmaxflap.dat};
%\end{axis}
%\end{tikzpicture}
%\caption{\footnotesize Profilo {\bfseries S4110}, variazione del $C_{l_{max}}$ con l 'angolo di deflessione del flap  per $ Re=1*10^6$}
%\end{figure}


\newpage
\section {Configurazione con Alettoni}

Per studiare il comportamento del profilo con Alettoni sono state previste deflessioni di  $+6^\circ $ e $-6^\circ $. Di seguito sono riportati i grafici del $C_P$ e delle curve di portanza nelle varie configurazioni. \\ 

\begin{figure} [H]
\centering
\begin{tikzpicture} 
\begin{axis} [ 
ylabel style={rotate=-90}, xmin=0, 
xmax=1, 
ymin=-0.1,
 ymax=0.1,
 xlabel=$ \frac {x}{c}$, 
ylabel=$ \frac {z}{c}$,
ytick={-0.1,-0.05,0,0.05,0.1},
yticklabels={$-0.1$,$-0.05$,$0$,$0.05$,$0.1$},
width=13cm,
 height=2.6 cm,
scale only axis,
grid=major] 
\addplot [black,solid,very thin]
file{images/fileDat/FlapsAlettoni/ProfiloPW106Chiuso.dat};
\addplot [black,solid,very thick]
file{images/fileDat/FlapsAlettoni/alettone_6deg.dat};
\end{axis}
\end{tikzpicture}\\

\begin{tikzpicture} 
\begin{axis} [ 
ylabel style={rotate=-90}, xmin=0, 
xmax=1, 
ymin=-0.1,
 ymax=0.1,
 xlabel=$ \frac {x}{c}$, 
ylabel=$ \frac {z}{c}$,
ytick={-0.1,-0.05,0,0.05,0.1},
yticklabels={$-0.1$,$-0.05$,$0$,$0.05$,$0.1$},
width=13cm,
 height=2.6 cm,
scale only axis,
grid=major] 
\addplot [black,solid,very thin]
file{images/fileDat/FlapsAlettoni/ProfiloPW106Chiuso.dat};
\addplot [black,solid,very thick]
file{images/fileDat/FlapsAlettoni/ProfiloAlettone_meno6Deg.dat};
\end{axis}
\end{tikzpicture}\\

\caption{\footnotesize Profilo alare PW106, configurazione con alettoni per $6^\circ$  $-6^\circ$ }
\end{figure}

\begin{figure} [H]
\centering
\begin{tikzpicture} 
\begin{axis} [ 
ylabel style={rotate=-90}, xmin=0, 
xmax=1, 
ymin=-2.5,
ymax=1,
xlabel=$\frac{x}{c}$, 
ylabel=$C_p$ ,
 y dir=reverse,
width=12cm,
height=6 cm,
scale only axis,
grid=major] 
\addplot [black,ultra thin]
file{images/fileDat/FlapsAlettoni/Cp_alfa_4_Re_1_10e6_dorso.dat};
\addplot [black,thick]
file{images/fileDat/FlapsAlettoni/cp_4deg_alettone_6deg_dorso.dat};
\addplot [black, ultra thick]
file{images/fileDat/FlapsAlettoni/cp_4deg_alettoneNegativo_dorso.dat};
\addplot [black, dashed, ultra thin]
file{images/fileDat/FlapsAlettoni/Cp_alfa_4_Re_1_10e6_ventre.dat};
\addplot [black, dashed, thick]
file{images/fileDat/FlapsAlettoni/cp_4deg_alettone_6deg_ventre.dat};
\addplot [black, dashed, ultra thick]
file{images/fileDat/FlapsAlettoni/cp_4deg_alettoneNegativo_ventre.dat};
\legend {$\delta_a=0 ^\circ$,$\delta_a=6 ^\circ$, $\delta_a=-6 ^\circ$}
\end{axis}
\end{tikzpicture}
\caption{\footnotesize Profilo alare PW106, confronto del Coefficiente di Pressione sul dorso del profilo (linea continua) e sul ventre (linea tratteggiata) in configurazione pulita e con alettoni in caso incomprimibile. $Re=1\times10^6$ $\alpha =4 ^\circ$ }
\end{figure}


\begin{figure} [H]
\centering
\begin{tikzpicture} 
\begin{axis} [ 
legend style={at={(0.3,0.98)}},
ylabel style={rotate=-90}, xmin=2, 
xmax=18, 
ymin=0,
ymax=1.5,
xlabel=${\alpha}$,
ylabel=$C_l$ ,
width=10cm,
height=12.5 cm,
scale only axis,
grid=major] 
\addplot [black, smooth, thick]
file{images/fileDat/FlapsAlettoni/clFlapClean.dat};
\addplot [black,smooth,mark=*, thick]
file{images/fileDat/FlapsAlettoni/Cl_re_106_alettone_6_deg.dat};
\addplot [black, smooth,mark=diamond*, thick]
file{images/fileDat/FlapsAlettoni/Cl_re_106_alettone_meno6_deg.dat};
\legend {$\delta_a=0 ^\circ$,$\delta_a=6 ^\circ$, $\delta_a=-6 ^\circ$}
\end{axis}
\end{tikzpicture} 
\caption{\footnotesize Profilo alare PW106, confronto delle curve di portanza in caso incomprimibile per $ Re=1\times10^6$ per alettoni deflessi e in configurazione pulita }
\end{figure}


\newpage
\section {Configurazione con {\itshape Droop Nose}}
Nell'ambito delle analisi del profilo alare PW106 con dispositivi mobili, è stata condotta un'ulteriore analisi con {\itshape droop nose}, un dispositivo di ipersostentazione da bordo d'attacco, simile allo {\itshape slat} ove l'intera sezione del bordo anteriore ruota verso il basso. Al fine di definire tale configurazione in xfoil, è stato utilizzato il comando {\itshape f} nel menu {\itshape GDES} di xfoil, ove è stata ruotata tutta la parte posteriore dell'angolo di deflessione del {\itshape droop nose} e poi il profilo è stato esso stesso ruotato e traslato con i comandi {\itshape ADEG} e {\itshape Tran} per riposizionare il TE in y=0. La traslazione lungo l'asse y è stata misurata digitalizzando il grafico e misurandone la distanza.
Per studiare il comportamento del profilo con {\itshape droop nose} è stata prevista una deflessione di  $\delta_{DN} = 15^\circ $ mentre la corda è stata scelta pari al 15\% della corda del profilo, i cui risultati principali sono riportati in  figura~\vref{fig:dn1},  figura~\vref{fig:dn2} e in  tabella~\vref{tab:droopn}.\\

\begin{figure} [H]
\centering
\begin{tikzpicture} 
\begin{axis} [ 
ylabel style={rotate=-90}, xmin=0, 
xmax=1, 
ymin=-0.1,
 ymax=0.1,
 xlabel=$ \frac {x}{c}$, 
ylabel=$ \frac {z}{c}$,
ytick={-0.1,-0.05,0,0.05,0.1},
yticklabels={$-0.1$,$-0.05$,$0$,$0.05$,$0.1$},
width=13cm,
 height=2.6 cm,
scale only axis,
grid=major] 
\addplot [black,solid,very thin]
file{images/fileDat/FlapsAlettoni/ProfiloPW106Chiuso.dat};
\addplot [black,solid,very thick]
file{images/fileDat/FlapsAlettoni/ProfiloDropNose15Deg.dat};
\end{axis}
\end{tikzpicture}
\caption{\footnotesize Profilo alare PW106, configurazione pulita e con {\itshape droop nose} per $\delta_{DN} = 15^\circ $}
\end{figure}


\begin{figure} [H]
\centering
\begin{tikzpicture} 
\begin{axis} [ 
ylabel style={rotate=-90}, xmin=0, 
xmax=1, 
ymin=-2.5,
ymax=1,
xlabel=$\frac{x}{c}$, 
ylabel=$C_p$ ,
 y dir=reverse,
width=9cm,
height=6 cm,
scale only axis,
grid=major] 
\addplot [black,ultra thin]
file{images/fileDat/FlapsAlettoni/Cp_alfa_4_Re_1_10e6_dorso.dat};
\addplot [black,thick]
file{images/fileDat/FlapsAlettoni/cp_re_106_alpha_4_droop_nose_15_deg_dorso.dat};
\addplot [black, dashed, ultra thin]
file{images/fileDat/FlapsAlettoni/Cp_alfa_4_Re_1_10e6_ventre.dat};
\addplot [black, dashed, thick]
file{images/fileDat/FlapsAlettoni/cp_re_106_alpha_4_droop_nose_15_deg_ventre.dat};
\legend {$\delta_{DN}=0 ^\circ$,$\delta_{DN} = 15^\circ $}
\end{axis}
\end{tikzpicture}
\caption{\footnotesize Profilo alare PW106, confronto del Coefficiente di Pressione sul dorso del profilo (linea continua) e sul ventre (linea tratteggiata) in configurazione pulita e con {\itshape droop nose} in caso incomprimibile. $Re=1\times10^6$ $\alpha =4 ^\circ$ }
\label{fig:dn1}
\end{figure}


\begin{table} [!h]\centering \rowcolors{1}{}{grigio_chiaro}
\begin{tabular}{c S c }
\toprule
\emph{Angolo di deflessione del {\itshape droop nose} } &  $C_{l_{\mathrm{max}}}$  &  ${\alpha}_{\mathrm{stall}}$  \\ 
\midrule
$\delta_{\mathrm{DN}}=0^\circ$ & 1.37 & $13.1^\circ$   \\
$\delta_{\mathrm{DN}}=15^\circ$ & 1.51 & $15.8^\circ$  \\
\bottomrule
\end{tabular}
\caption {Profilo alare PW106, $C_{l_{\mathrm{max}}}$ e ${\alpha}_{\mathrm{stall}}$ per {\itshape droop nose} retratto e deflesso, $ Re=1\times10^6$.}
\label{tab:droopn}
\end{table}


\begin{figure} [H]
\centering
\begin{tikzpicture} 
\begin{axis} [ 
legend style={at={(0.3,0.98)}},
ylabel style={rotate=-90}, xmin=2, 
xmax=18, 
ymin=0,
ymax=1.6,
xlabel=${\alpha}$,
ylabel=$C_l$ ,
width=10cm,
height=12.5 cm,
scale only axis,
grid=major] 
\addplot [black, smooth, thick]
file{images/fileDat/FlapsAlettoni/clFlapClean.dat};
\addplot [black,ultra thick,mark=*]
file{images/fileDat/FlapsAlettoni/Curva_portanze_re106_droopNose_15deg.dat};
\legend {$\delta_{DN}=0 ^\circ$,$\delta_{DN} = 15^\circ $}
\end{axis}
\end{tikzpicture} 
\caption{\footnotesize Profilo alare PW106, confronto delle curve di portanza in caso incomprimibile per $ Re=1\times10^6$ con {\itshape droop nose} deflesso e in configurazione pulita }
\label{fig:dn2}
\end{figure}

Infine è stato riportato un caso con contemporanea deflessione di flap a $\delta_{f} = 25^\circ $ e {\itshape droop nose} a $\delta_{DN} = 15^\circ$, i cui risultati sono riportati in  figura~\vref{fig:dn3},  figura~\vref{fig:dn4} e in tabella~\vref{tab:droopn1}.

\begin{figure} [H]
\centering
\begin{tikzpicture} 
\begin{axis} [ 
ylabel style={rotate=-90}, xmin=0, 
xmax=1, 
ymin=-0.2,
 ymax=0.1,
 xlabel=$ \frac {x}{c}$, 
ylabel=$ \frac {z}{c}$,
ytick={-0.2,-0.1,0,0.1},
yticklabels={$-0.2$,$-0.1$,$0$,$0.1$},
width=13cm,
 height=2.6 cm,
scale only axis,
grid=major] 
\addplot [black,solid,very thin]
file{images/fileDat/FlapsAlettoni/ProfiloPW106Chiuso.dat};
\addplot [black,solid,very thick]
file{images/fileDat/FlapsAlettoni/ProfiloConFlapEDN.dat};
\end{axis}
\end{tikzpicture}
\caption{\footnotesize Profilo alare PW106, configurazione pulita e con {\itshape droop nose} per $\delta_{DN} = 15^\circ $}
\end{figure}

\begin{figure} [H]
\centering
\begin{tikzpicture} 
\begin{axis} [ 
ylabel style={rotate=-90}, xmin=0, 
xmax=1, 
ymin=-4,
ymax=1,
xlabel=$\frac{x}{c}$, 
ylabel=$C_p$ ,
 y dir=reverse,
width=12cm,
height=6 cm,
scale only axis,
grid=major] 
\addplot [black,ultra thin]
file{images/fileDat/FlapsAlettoni/Cp_alfa_4_Re_1_10e6_dorso.dat};
\addplot [black,solid,very thick]
file{images/fileDat/FlapsAlettoni/Cp_Flap_DN_4degdorso.dat};
\addplot [black, dashed, ultra thin]
file{images/fileDat/FlapsAlettoni/Cp_alfa_4_Re_1_10e6_ventre.dat};
\addplot [black, dashed, thick]
file{images/fileDat/FlapsAlettoni/Cp_Flap_DN_4degventre.dat};
\legend {configurazione pulita ,$\delta_{DN} = 15^\circ$  e  $\delta_{f} = 25^\circ $}
\end{axis}
\end{tikzpicture}
\caption{\footnotesize Profilo alare PW106, confronto del Coefficiente di Pressione sul dorso del profilo (linea continua) e sul ventre (linea tratteggiata) in configurazione pulita e con {\itshape droop nose} e {\itshape flap} in caso incomprimibile. $Re=1\times10^6$ $\alpha =4 ^\circ$ }
\label{fig:dn3}
\end{figure}
\noindent \\ 
\begin{table} [!h]\centering \rowcolors{1}{}{grigio_chiaro}
\begin{tabular}{c S c }
\toprule
\emph{Angolo di deflessione dele superfici mobili } &  $C_{l_{\mathrm{max}}}$  &  ${\alpha}_{\mathrm{stall}}$  \\ 
\midrule
Configurazione pulita & 1.37 & $13.1^\circ$   \\
$\delta_{\mathrm{DN}}=15^\circ$  e $\delta_{flap}=25 ^\circ$ & 1.94 & $9.5^\circ$  \\
\bottomrule 
\end{tabular}
\caption {Profilo alare PW106, $C_{l_{\mathrm{max}}}$ e ${\alpha}_{\mathrm{stall}}$ per configurazione pulita e {\itshape droop nose} e {\itshape flap} deflessi, $ Re=1\times10^6$.}
\label{tab:droopn1}
\end{table}
\noindent \\ \\

\begin{figure} [H]
\centering
\begin{tikzpicture} 
\begin{axis} [ 
legend style={at={(0.5,0.98)}},
ylabel style={rotate=-90}, xmin=-5, 
xmax=18, 
ymin=0,
ymax=2.4,
xlabel=${\alpha}$,
ylabel=$C_l$ ,
width=10cm,
height=13 cm,
scale only axis,
grid=major] 
\addplot [black, smooth, thick]
file{images/fileDat/FlapsAlettoni/clFlapClean.dat};
\addplot [black,smooth,mark=*, thick]
file{images/fileDat/FlapsAlettoni/ClFlap25.dat};
\addplot [black, smooth,mark=diamond*, thick]
file{images/fileDat/FlapsAlettoni/Curva_portanze_re106_droopNose_15deg.dat};
\addplot [black, smooth,mark=square, thick]
file{images/fileDat/FlapsAlettoni/Curva_portanze_re106_droopNose_15deg_flap25.dat};
\legend {configurazione pulita ,$\delta_f=25 ^\circ$, $\delta_{DN}=15 ^\circ$, $\delta_f=25 ^\circ$ e $\delta_{DN}=15 ^\circ$}
\end{axis}
\end{tikzpicture} 
\caption{\footnotesize Profilo alare PW106, confronto delle curve di portanza in caso incomprimibile per $ Re=1\times10^6$ in configurazione pulita, con flap deflessi, con {\itshape droop nose} deflesso e con entrambi  }
\label{fig:dn4}
\end{figure}

