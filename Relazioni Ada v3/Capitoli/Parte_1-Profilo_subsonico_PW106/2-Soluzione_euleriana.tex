% !TeX program = PdfLaTeX
% !TeX root = ../../Elaborati_Aerodinamica_Bruno_Spoti.tex
\chapter{Aerodinamica non viscosa incomprimibile}
Nel seguente capitolo verranno esposti i risultati ottenuti circa le caratteristiche del profilo in termini di retta di portanza, angolo d’attacco ideale e di portanza nulla e coefficiente di momento focale. Infine verranno graficate le distribuzioni del coefficiente di pressione al variare del $C_l $.
\section{Curve di portanza e momento}
Tramite l’utilizzo di XFOIL è stato possibile ottenere l’andamento della curva di portanza del profilo PW106 e quello del momento aerodinamico in termini di soluzione Euleriana incomprimibile, calcolato rispetto al punto ad un quarto della corda. Al fine di calcolare il centro aerodinamico è stata applicata la formula~\vref{eq:centroAerodinamico}, dalla particolarizzazione della quale è stata ricavata anche la curva del momento rispetto il bordo d'attacco riportate in figura~\vref{fig:cmIncomp}. I principali risultati sono riportati in tabella~\vref{tab:risultatiXFOIL}.


\renewcommand\arraystretch{1.4} 
\begin{table} [!h]\centering \rowcolors{1}{}{grigio_chiaro}
\begin{tabular}{c| c}
\toprule
  &  \emph{XFOIL 6.99} \\ 
\midrule
${\alpha}_{\mathrm{zl}}$ & $-0.872^\circ$ \\
${C_{l_{\alpha}}}$ & $0.117 \, \si{deg}^{-1}$ \\
${ C_{m_{{\alpha}_{c4}}}}$ &$-0.0007 \, \si{deg}^{-1}$  \\ 
${ C_{m_{{\alpha}_{le}}}}$ &$-0.0301 \, \si{deg}^{-1}$\\
 $\frac{X_{ac}}{c}$  & 0.256\\
\bottomrule
\end{tabular}
\caption {Profilo alare PW106, principali risultati ottenuti tramite XFOIL 6.99.}
\label{tab:risultatiXFOIL}
\end{table}

\begin{equation}
\frac {x_{ac}}{c}=\frac {x_{c_4}}{c} - \frac{ C_{m_{{\alpha}_{c4}}}}{C_{l_{\alpha}}}
\label{eq:centroAerodinamico}
\end{equation}


\begin{figure} [H]
\centering
\begin{tikzpicture} 
\begin{axis} [ 
ylabel style={rotate=-90}, xmin=-10, 
xmax=10, 
ymin=-1,
ymax=1.6,
xlabel=$ \alpha$, 
ylabel=$ C_l $,
ytick={-1,-0.8,-0.6,-0.4,-0.2,0,0.2,0.4,0.6,0.8,1,1.2,1.4,1.6},
width=9cm,
height=7.5cm,
scale only axis,
grid=major] 
\addplot [black]
file{images/fileDat/AnalisiNonViscoseIncomprimibili/Cl_vs_alpha.dat};
\end{axis}
\end{tikzpicture}
\caption{\footnotesize Profilo alare PW106, curva di portanza, soluzione euleriana incomprimibile.  XFOIL 6.99 }\label{fig:cl}
\end{figure}
\noindent
\\ 

\begin{figure} [h!]
\centering
\begin{tikzpicture} 
\begin{axis} [ 
ylabel style={rotate=-90}, xmin=-10, 
xmax=10, 
ymin=-0.35,
ymax=0.3,
xlabel=$ \alpha$, 
ylabel=$ C_{m}$,
%ytick={-0.5,-0.4,-0.3,-0.2,-0.1,0,0.1,0.2,0.3,0.4,0.5},
width=9cm,
height=7.5 cm,
scale only axis,
grid=major] 
\addplot [black]
file{images/fileDat/AnalisiNonViscoseIncomprimibili/Cm_c_quarti_vs_alpha_v2.dat};
\addplot [black,dashed]
file{images/fileDat/AnalisiNonViscoseIncomprimibili/Cm_le_vs_alpha_v2.dat};
\addplot [black,dashdotted]
file{images/fileDat/AnalisiNonViscoseIncomprimibili/Cm_ac_vs_alpha_v2.dat};
\legend {${ C_{m_{c4}}}$, ${ C_{m_{le}}}$ , ${ C_{m_{ac}}}$}
\end{axis}
\end{tikzpicture}
\caption{\footnotesize Profilo alare PW106, curve di momento, soluzione euleriana incomprimibile al variare del polo.  XFOIL 6.99 }\label{fig:cmIncomp}
\end{figure}



\noindent \\
\section{Centro di Pressione}

Il centro di pressione é quel punto in cui si puó ritenere applicata la risultante delle forze fluidodinamiche agenti sul profilo.
Per calcolarlo si é ricorsi alla formula~\vref{eq:centroPressione}. \cite{prof:tognaccini}

\begin{equation}
-\frac {x_{cp}}{c}=\frac{ C_{m_{le}}}{C_l}
\label{eq:centroPressione}
\end{equation}

Il $ C_{m_{le}}$ è il coefficiente di momento calcolato rispetto al bordo d’attacco del profilo. \\
Esso può essere ottenuto tramite il calcolo dei coefficienti di Fourier dello sviluppo in serie della derivata della linea media, in accordo con la Teoria del Profilo Sottile, i cui risultati saranno in seguito esposti.

\begin{equation}
\label{eqn:zerouno}
C_{m_{le}}= \frac {{\pi}}{4}(c_2-c_1)- \frac {C_l}{4}
\end{equation}

\begin{figure} [h!]
\centering
\begin{tikzpicture} 
\begin{axis} [ 
ylabel style={rotate=-90}, xmin=-3, 
xmax=2, 
ymin=-0.25,
ymax=0.8,
xlabel=$ {\alpha}$, 
ylabel=$\frac{x_{cp}}{c}$,
width=9cm,
height=7 cm,
scale only axis,
grid=major] 
\addplot [black,smooth]
file{images/fileDat/AnalisiNonViscoseIncomprimibili/XcpCAlpha1.dat};
\addplot [black,smooth]
file{images/fileDat/AnalisiNonViscoseIncomprimibili/XcpCAlpha2.dat};
\end{axis}
\end{tikzpicture}
\caption{\footnotesize Profilo alare PW106, andamento dell'ascissa del centro di pressione al variare di ${\alpha}$.  XFOIL 6.99}\label{fig:cp}
\end{figure}


\section{Risultati della Teoria del Profilo Sottile}

Al fine di calcolare i coefficienti di Fourier dello sviluppo in serie della derivata della linea media, è stato implementato uno script in MATLAB che, ricevendo in ingresso i punti della linea media, consente di ottenere il gradiente della retta di portanza, l’angolo d’attacco ideale e di portanza nulla e il coefficiente di momento di beccheggio rispetto al fuoco.

I coefficienti dello sviluppo in serie di Fourier valgono\\

\begin{equation}
\label{eqn:prima}
c_0= \frac {1}{{\pi}}\int_0^{\pi} C'(x) d{\theta}
\end{equation}

\begin{equation}
\label{eqn:seconda}
c_n= \frac {2}{{\pi}}\int_0^{\pi} C'(x) {\cos}(n{\theta}) d{\theta}
\end{equation}


\noindent \\ \\
Dalle formule \ref{eqn:prima} e \ref{eqn:seconda} sono stati ricavati i seguenti risultati, i cui valori sono riportati in 


\begin {itemize}
\item ${\alpha}_i=c_0$
\item ${\alpha}_{zl}=c_0-\frac {c_1}{2}$
\item $C_{m\frac {c}{4}}=-\frac {{\pi}}{4}(c_1-c_2)$
\end{itemize}

\noindent \\ 
Per il calcolo è stato utilizzato il $C_{l{\alpha}}$ ottenuto con XFOIL pari a $6.73 rad^{-1}$.

%\begin{equation}
%\label{eqn:terza}
%c_{l_{\alpha}}=2{\pi}(1+ k{\tau})
%\end{equation}

%\noindent \\
%Il cui risultato è \\
%
%\begin{center}
% { k=0.79}
%\end {center}


\renewcommand\arraystretch{1.4} 
\begin{table} [!h]\centering \rowcolors{1}{}{grigio_chiaro}
\begin{tabular}{c| c}
\toprule
  &  \emph{Teoria del profilo sottile} \\ 
\midrule
${\alpha}_{i} $ & $0.0363^\circ$ \\
${\alpha}_{\mathrm{zl}}$ & $-0.789^\circ$ \\
${ C_{m_{{\alpha}_{c4}}}}$ &$-0.00103$  \\ 
\bottomrule
\end{tabular}
\caption {Profilo alare PW106, principali risultati ottenuti tramite l'applicazione della teoria del profilo sottile implementata mediante MATLAB R2016b.}
\label{tab:profiloSottile}
\end{table}

\section{Coefficiente di Pressione}

Di seguito sono riportati gli andamenti del coefficiente di pressione del profilo alare in studio, nell’ ipotesi di campo Euleriano incomprimibile, in funzione dell’ascissa adimensionalizzata rispetto la corda a vari $C_l$, anche non piccoli.\\ \\


\begin{figure} [h!]
\centering
\begin{tikzpicture} 
\begin{axis} [ 
ylabel style={rotate=-90}, xmin=0, 
xmax=1, 
ymin=-17,
ymax=2,
xlabel=$\frac{x}{c}$, 
ylabel=$C_p$ ,
 y dir=reverse,
width=12cm,
height=7 cm,
scale only axis,
grid=major] 
\addplot [black]
file{images/fileDat/AnalisiNonViscoseIncomprimibili/cp_-1_Inv_Inc_dorso.dat};
\addplot [black, dashed]
file{images/fileDat/AnalisiNonViscoseIncomprimibili/cp_-1_Inv_Inc_ventre.dat};
\legend {Dorso,Ventre}
\end{axis}
\end{tikzpicture}
\caption{\footnotesize Profilo alare PW106, distribuzione del coefficiente di pressione $C_l=-1 \  ( \alpha=-9.40^\circ$). Soluzione Euleriana incomprimibile.  XFOIL 6.99 }\label{fig:cp}
\end{figure}

\begin{figure} 
\centering
\begin{tikzpicture} 
\begin{axis} [ 
ylabel style={rotate=-90}, xmin=0, 
xmax=1, 
ymin=-1.5,
ymax=1,
xlabel=$\frac{x}{c}$, 
ylabel=$C_p$ ,
 y dir=reverse,
width=12cm,
height=7 cm,
scale only axis,
grid=major] 
\addplot [black]
file{images/fileDat/AnalisiNonViscoseIncomprimibili/cp_0_Inv_Inc_dorso.dat};
\addplot [black, dashed]
file{images/fileDat/AnalisiNonViscoseIncomprimibili/cp_0_Inv_Inc_ventre.dat};
\legend {Dorso,Ventre}
\end{axis}
\end{tikzpicture}
\caption{\footnotesize Profilo alare PW106, distribuzione del coefficiente di pressione $C_l=0 \ ( \alpha \!= \! -0.873^\circ$). Soluzione Euleriana incomprimibile.  XFOIL 6.99 }\label{fig:cp}
\end{figure}


\begin{figure} 
\centering
\begin{tikzpicture} 
\begin{axis} [ 
ylabel style={rotate=-90}, xmin=0, 
xmax=1, 
ymin=-1.2,
ymax=1,
xlabel=$\frac{x}{c}$, 
ylabel=$C_p$ ,
 y dir=reverse,
width=12cm,
height=7 cm,
scale only axis,
grid=major] 
\addplot [black]
file{images/fileDat/AnalisiNonViscoseIncomprimibili/cp_0.5_Inv_Inc_dorso.dat};
\addplot [black, dashed]
file{images/fileDat/AnalisiNonViscoseIncomprimibili/cp_0.5_Inv_Inc_ventre.dat};
\legend {Dorso,Ventre}
\end{axis}
\end{tikzpicture}
\caption{\footnotesize Profilo alare PW106, distribuzione del coefficiente di pressione $C_l=0.5 \ ( \alpha=3.39^\circ )$. Soluzione Euleriana incomprimibile.  XFOIL 6.99 }\label{fig:cp}
\end{figure}
\noindent \\


\begin{figure} 
\centering
\begin{tikzpicture} 
\begin{axis} [ 
ylabel style={rotate=-90}, xmin=0, 
xmax=1, 
ymin=-3.5,
ymax=1.5,
xlabel=$\frac{x}{c}$, 
ylabel=$C_p$ ,
 y dir=reverse,
width=12cm,
height=7 cm,
scale only axis,
grid=major] 
\addplot [black]
file{images/fileDat/AnalisiNonViscoseIncomprimibili/cp_1_Inv_Inc_dorso.dat};
\addplot [black, dashed]
file{images/fileDat/AnalisiNonViscoseIncomprimibili/cp_1_Inv_Inc_ventre.dat};
\legend {Dorso,Ventre}
\end{axis}
\end{tikzpicture}
\caption{\footnotesize Profilo alare PW106, distribuzione del coefficiente di pressione $C_l=1 \ ( \alpha=7.68^\circ$). Soluzione Euleriana incomprimibile.  XFOIL 6.99 }\label{fig:cp}
\end{figure}
\noindent \\


\begin{figure} 
\centering
\begin{tikzpicture} 
\begin{axis} [ 
ylabel style={rotate=-90}, xmin=0, 
xmax=1, 
ymin=-9,
ymax=1.5,
xlabel=$\frac{x}{c}$, 
ylabel=$C_p$ ,
 y dir=reverse,
width=12cm,
height=7 cm,
scale only axis,
grid=major] 
\addplot [black]
file{images/fileDat/AnalisiNonViscoseIncomprimibili/cp_1.4_Inv_Inc_dorso.dat};
\addplot [black, dashed]
file{images/fileDat/AnalisiNonViscoseIncomprimibili/cp_1.4_Inv_Inc_ventre.dat};
\legend {Dorso,Ventre}
\end{axis}
\end{tikzpicture}
\caption{\footnotesize Profilo alare PW106, distribuzione del coefficiente di pressione $C_l=1.4 \ ( \alpha=10.3^\circ$). Soluzione Euleriana incomprimibile.  XFOIL 6.99 }\label{fig:cp}
\end{figure}
\noindent 




