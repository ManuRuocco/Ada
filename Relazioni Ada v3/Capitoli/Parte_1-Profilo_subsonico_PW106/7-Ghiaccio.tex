% !TeX program = PdfLaTeX
% !TeX root = ../../Elaborati_Aerodinamica_Bruno_Spoti.tex
\chapter{Effetti del ghiaccio}
%L'accrescimento di ghiaccio sulle superfici dei velivoli può essere studiato come un cambiamento di geometria che genera un degrado delle prestazioni aerodinamiche o di manovrabilità del velivolo. In particolare ciò si traduce in un degrado del $C_{l_{\mathrm{max}}}$  e dell' ${\alpha}_{\mathrm{stall}}$ con un conseguente aumento della velocità di stallo, e un incremento del coefficiente di resistenza. 
%Di seguito sono state riportate le analisi per due diverse modalità di accrescimento del ghiaccio: {\itshape Rime icing} ove tutte le goccioline di acqua che impattano si ghiacciano, e {\itshape Glaze icing}, caratterizzato da forme più complesse in quanto c'è una frazione d'acqua on ghiacciata.
%Considerato il profilo di interesse, il PW106, e la sua applicabilità su velivoli non di grandi dimensioni, tale problematicità può avere un peso notevole vista la dimensione relativa tra ghiaccio e profilo stesso. 
%Al fine di disegnare il nuovo profilo con l'opportuna modifica al bordo d'attacco, i punti assegnati sono stati importati in un software per digitalizzare grafici |ref{sito:plotdigitizer}, ed è stato disegnato per punti, il nuovo bordo d'attacco con accrescimento di ghiaccio. Il profilo così definito è stato poi importato in XFOIL 6.99 per le analisi aerodinamiche.
L'accrescimento di ghiaccio sulle superfici dei velivoli può essere studiato come un cambiamento di geometria che genera un degrado delle prestazioni aerodinamiche o di manovrabilità del velivolo. In particolare ciò si traduce in un degrado del $C_{l_{\mathrm{max}}}$  e dell' ${\alpha}_{\mathrm{stall}}$ con un conseguente aumento della velocità di stallo, e un incremento del coefficiente di resistenza. 
Di seguito sono state riportate i profili per due diverse modalità di accrescimento del ghiaccio: {\itshape Rime icing} ove tutte le goccioline di acqua che impattano si ghiacciano, e {\itshape Glaze icing}, caratterizzato da forme più complesse in quanto c'è una frazione d'acqua on ghiacciata.
Considerato il profilo di interesse, il PW106, e la sua applicabilità su velivoli non di grandi dimensioni, tale problematicità può avere un peso notevole vista la dimensione relativa tra ghiaccio e profilo stesso. 
Al fine di disegnare il nuovo profilo con l'opportuna modifica al bordo d'attacco, i punti assegnati sono stati importati in un software per digitalizzare grafici |ref{sito:plotdigitizer}, ed è stato disegnato per punti, il nuovo bordo d'attacco con accrescimento di ghiaccio. Il profilo così definito è stato poi importato in XFOIL 6.99 per le analisi aerodinamiche. Le analisi sono state svolte sulla configurazione con {\itshape Rime icing}, in quanto a causa dell'elevata spigolosità dell caso  {\itshape Glaze icing}, il solutore XFOIL non è in grado di portare a convergenza la soluzione.

\begin{figure}
\centering
\begin{tikzpicture} 
\begin{axis} [ 
ylabel style={rotate=-90}, xmin=-0.1, 
xmax=1.1, 
ymin=-0.1,
 ymax=0.1,
 xlabel=$ \frac {x}{c}$, 
ylabel=$ \frac {z}{c}$,
ytick={-0.1,-0.05, 0,0.05, 0.1},
yticklabels={$-0.1$,$-0.05$,$0$,$-0.05$,$0.1$},
width=13cm,
 height=2.1667 cm,
scale only axis,
grid=major] 
\addplot [black,solid,very thick]
file{images/fileDat/Ghiaccio/ProfiloPW106Chiuso.dat};
\addplot [black,solid,thin]
file{images/fileDat/Ghiaccio/ProfiloPW106_Rime_icing.dat};
\end{axis}
\end{tikzpicture}
\caption{\footnotesize Profilo alare PW106, configurazione pulita e con accrescimento di ghiaccio di tipo {\itshape Rime icing}}
\label{fig:rimeiicing}
\end{figure}

\begin{figure}
\centering
\begin{tikzpicture} 
\begin{axis} [ 
ylabel style={rotate=-90}, xmin=-0.1, 
xmax=1.1, 
ymin=-0.1,
 ymax=0.1,
 xlabel=$ \frac {x}{c}$, 
ylabel=$ \frac {z}{c}$,
ytick={-0.1,-0.05, 0,0.05, 0.1},
yticklabels={$-0.1$,$-0.05$,$0$,$-0.05$,$0.1$},
width=13cm,
 height=2.1667 cm,
scale only axis,
grid=major] 
\addplot [black,solid,very thick]
file{images/fileDat/Ghiaccio/ProfiloPW106Chiuso.dat};
\addplot [black,solid,thin]
file{images/fileDat/Ghiaccio/ProfiloPW106_Glaze_icing.dat};
\end{axis}
\end{tikzpicture}
\caption{\footnotesize Profilo alare PW106, configurazione pulita e con accrescimento di ghiaccio di tipo {\itshape Glaze icing}}
\label{fig:glazi}
\end{figure}


\begin{figure} [H]
\centering
\begin{tikzpicture} 
\begin{axis} [ 
legend style={at={(0.45,0.98)}},
ylabel style={rotate=-90}, xmin=0, 
xmax=16, 
ymin=0,
ymax=1.4,
xlabel=${\alpha}$,
ylabel=$C_l$ ,
width=12cm,
height=14 cm,
scale only axis,
grid=major] 
\addplot [black, smooth,mark=*]
file{images/fileDat/Ghiaccio/liftCurve_Pulito.dat};
\addplot [black,smooth,mark=square]
file{images/fileDat/Ghiaccio/liftCurve_RE25e4_rime_icing.dat};
\legend {configurazione pulita ,Rime icing}
\end{axis}
\end{tikzpicture} 
\caption{\footnotesize Profilo alare PW106, confronto delle curve di portanza in configurazione pulita e con formazione di ghiacccio, soluzione viscosa con $ Re= 2.5\times10^5$. XFOIL 6.99 }
\label{fig:clicing}
\end{figure}

\begin{figure} [H]
\centering
\begin{tikzpicture} 
\begin{axis} [ 
legend style={at={(0.98,0.30)}},
ylabel style={rotate=-90}, xmin=0, 
ymin=0.2,
ymax=1.4,
xlabel=$C_d$ (drag count),
ylabel=$C_l$ ,
width=11cm,
height=11cm,
scale only axis,
grid=major] 
\addplot [black, smooth,mark=*]
file{images/fileDat/Ghiaccio/polarCurve_Pulito.dat};
\addplot [black,smooth,mark=square]
file{images/fileDat/Ghiaccio/PolarCurve_RE25e4_rime_icing.dat};
\legend {configurazione pulita ,Rime icing}
\end{axis}
\end{tikzpicture}
\caption{\footnotesize Profilo alare PW106, confronto delle polari in configurazione pulita e con formazione di ghiacccio, soluzione viscosa con $ Re= 2.5\times10^5$. XFOIL 6.99 }
\label{fig:policing}
\end{figure}


\begin{table} [!h]\centering \rowcolors{1}{}{grigio_chiaro}
\begin{tabular}{c S c c}
\toprule
\emph{Configurazione} &  $C_{{l}_{ \alpha = 10^{\circ}}} $  &  $C_{{d}_{ \alpha = 10^{\circ}}}$ & $C_{l_{\mathrm{max}}}$  \\ 
\midrule
Configurazione pulita & 1.15 & $21.2$ & $1.21$  \\
{\itshape Rime icing} & 1.12 & $36.3$ & $1.17$  \\
\bottomrule
\end{tabular}
\caption {Profilo alare PW106, $C_{l_{\mathrm{max}}}$, ${\alpha}_{\mathrm{stall}}$ e ${\alpha}_{\mathrm{zl}}$ per flap retratto e deflesso, $ Re=1\times10^6$.}
\label{tab:flap}
\end{table}

