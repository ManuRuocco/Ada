\chapter{Aerodinamica non viscosa incomprimibile}

Nel seguente capitolo verranno esposti i risultati ottenuti circa le caratteristiche del profilo in termini di retta di portanza, angolo d’attacco ideale e di portanza nulla e coefficiente di momento focale. Infine verranno graficate le distribuzioni del coefficiente di pressione al variare del $C_l $.\\
Saranno di seguito rappresentati solo i risultati grafici in quanto le considerazioni relative alla costruzione di tali grafici sono già state fatte nella parte I.

\section{ Andamento di portanza e momento}

\begin{figure} [h!]
\centering
\begin{tikzpicture} 
\begin{axis} [ 
xmin=-10, 
xmax=10, 
ymin=-1,
ymax=1.6,
xlabel=$ \alpha$, 
ylabel=$ C_l $,
ytick={-1,-0.8,-0.6,-0.4,-0.2,0,0.2,0.4,0.6,0.8,1,1.2,1.4,1.6},
width=10cm,
height=7.5cm,
scale only axis,
grid=major] 
\addplot [black,very thick]
file{immagini/cla_laminare.dat};
\end{axis}
\end{tikzpicture}
\caption{\footnotesize Retta di portanza del profilo alare NACA $66_4-221$, soluzione euleriana incomprimibile }
\end{figure}
\noindent
\\ 

\begin{figure} [h!]
\centering
\begin{tikzpicture} 
\begin{axis} [ 
xmin=-10, 
xmax=10, 
ymin=-0.5,
ymax=0.5,
xlabel=$ \alpha$, 
ylabel=$ C_m$,
ytick={-0.5,-0.4,-0.3,-0.2,-0.1,0,0.1,0.2,0.3,0.4,0.5},
width=10cm,
height=6.5 cm,
scale only axis,
grid=major] 
\addplot [black,very thick]
file{immagini/cm_laminare.dat};
\end{axis}
\end{tikzpicture}
\caption{\footnotesize Curva di momento del profilo alare NACA $66_4-221$, soluzione euleriana incomprimibile, polo ad $ \frac {x}{c}$ =$ \frac {1}{4}$ }
\end{figure}

\section{Centro di Pressione}

Il centro di pressione è quel punto in cui si puó ritenere applicata la risultante delle forze fluidodinamiche agenti sul profilo.\\

\begin{figure} [H]
\centering
\begin{tikzpicture}
\begin{axis}[
xmin=-12,
xmax=12,
ymin=-4,
ymax=4,
xlabel=${\alpha}$,
ylabel=$\frac{x_{cp}}{c}$,
width=9cm,
height=6.5cm,
scale only axis,
grid=major]
\addplot[black, very thick]
file{immagini/xcp1_lam.dat};
\addplot[black, very thick]
file{immagini/xcp2_lam.dat};
\end{axis}
\end{tikzpicture}
\caption{\footnotesize Profilo S4110, Andamento dell'ascissa del Centro di Pressione al variare di ${\alpha}$ }\label{fig:cp}
\end{figure}


\section{Risultati della Teoria del Profilo Sottile}

Sono stati calcolati i coefficienti di Fourier dello sviluppo in serie della derivata della linea media ottenendo i seguenti risultati


\begin {itemize}
\item ${\alpha}_i=c_0=-0.0215^\circ$
\item ${\alpha}_{zl}=\frac {c_1}{2}-c_0=-1.69^\circ$
\item $C_{m\frac {c}{4}}=-\frac {{\pi}}{4}(c_1-c_2)=-0.0439$
\end{itemize}

\noindent \\ 
Ottenuto un $c_{l{\alpha}} = 7.16 rad^{-1}$  è possibile ricavare $ k=0.665$

\noindent \\
\section{Coefficiente di Pressione}

Di seguito sono riportati gli andamenti del coefficiente di pressione del profilo alare in studio, nell’ ipotesi di campo Euleriano incomprimibile, in funzione dell’ascissa adimensionalizzata rispetto alla corda a vari $C_l$ di particolare interesse per questo profilo laminare.


\begin{figure} [h!]
\centering
\begin{tikzpicture} 
\begin{axis} [ 
xmin=0, 
xmax=1, 
ymin=-1,
ymax=1,
xlabel=$\frac{x}{c}$, 
ylabel=$C_p$ ,
 y dir=reverse,
width=12cm,
height=7 cm,
scale only axis,
grid=major] 
\addplot [black,very thick, smooth]
file{immagini/CLMENO02.dat};
\end{axis}
\end{tikzpicture}
\caption{\footnotesize NACA $66_4-221$, distribuzione del Coefficiente di Pressione $C_l=-0.2$ (estremo inferiore sacca laminare). Soluzione Euleriana incomprimibile }
\end{figure}

Si noti come cambia notevolmente il comportamento del profilo in sacca e fuori la sacca laminare. Per un $C_l$ che non appartiene alla zona della sacca laminare, si ha un picco sul grafico del Cp mentre per l'intervallo di $C_l$ di sacca si ha assenza di picchi di espansione al bordo d'attacco il che assicura un gradiente di pressione favorevole con un esteso deflusso laminare.

\begin{figure} [H]
\centering
\begin{tikzpicture} 
\begin{axis} [ 
xmin=0, 
xmax=1, 
ymin=-1,
ymax=1,
xlabel=$\frac{x}{c}$, 
ylabel=$C_p$ ,
 y dir=reverse,
width=12cm,
height=7 cm,
scale only axis,
grid=major] 
\addplot [black,very thick,smooth]
file{immagini/cl_0_laminare.dat};
\end{axis}
\end{tikzpicture}
\caption{\footnotesize NACA $66_4-221$, distribuzione del Coefficiente di Pressione $C_l=0$. Soluzione Euleriana incomprimibile }
\end{figure}


\begin{figure} [h!]
\centering
\begin{tikzpicture} 
\begin{axis} [ 
xmin=0, 
xmax=1, 
ymin=-1,
ymax=1,
xlabel=$\frac{x}{c}$, 
ylabel=$C_p$ ,
 y dir=reverse,
width=12cm,
height=7 cm,
scale only axis,
grid=major] 
\addplot [black,very thick, smooth]
file{immagini/cl_0.2_laminare.dat};
\end{axis}
\end{tikzpicture}
\caption{\footnotesize NACA $66_4-221$, distribuzione del Coefficiente di Pressione $C_l=0.2$ (centro sacca). Soluzione Euleriana incomprimibile }
\end{figure}




\begin{figure} 
\centering
\begin{tikzpicture} 
\begin{axis} [ 
xmin=0, 
xmax=1, 
ymin=-1.5,
ymax=1,
xlabel=$\frac{x}{c}$, 
ylabel=$C_p$ ,
 y dir=reverse,
width=12cm,
height=7 cm,
scale only axis,
grid=major] 
\addplot [black,very thick, smooth]
file{immagini/cl_0.6_laminare.dat};
\end{axis}
\end{tikzpicture}
\caption{\footnotesize NACA $66_4-221$, distribuzione del Coefficiente di Pressione $C_l=0.6$ (estremo superiore sacca laminare). Soluzione Euleriana incomprimibile }
\end{figure}

\begin{figure} 
\centering
\begin{tikzpicture} 
\begin{axis} [ 
xmin=0, 
xmax=1, 
ymin=-3,
ymax=1,
xlabel=$\frac{x}{c}$, 
ylabel=$C_p$ ,
 y dir=reverse,
width=12cm,
height=7 cm,
scale only axis,
grid=major] 
\addplot [black,very thick]
file{immagini/cl_1_laminare.dat};
\end{axis}
\end{tikzpicture}
\caption{\footnotesize NACA $66_4-221$, distribuzione del Coefficiente di Pressione $C_l=1$ . Soluzione Euleriana incomprimibile }
\end{figure}

\noindent \\

