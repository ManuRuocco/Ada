% !TeX program = PdfLaTeX
% !TeX root = ../../Elaborati_Aerodinamica_Bruno_Spoti.tex
\chapter{Aerodinamica non viscosa del profilo alare alle alte velocità di volo}
Scopo della seguente parte è lo studio dell'aerodinamica di tipo non viscoso per un profilo alare alle alte velocità di volo. La geometria del profilo è stata fornita tramite le coordinate di quattro punti che rappresentano i vertici di un quadrilatero.\\
Lo studio del profilo è stato condotto attraverso uno {\itshape script} interamente scritto in MATLAB parametrizzato per le coordinate che ne definiscono la geometria e le condizioni operative. In primo luogo viene calcolata la geometria completa in termini di angoli interni e distanze tra i punti. Noti tali valori è valutato il $M'_{\infty\mathrm{cr}}$ in funzione dell'angolo d'attacco $\alpha$. Successivamente per $M_{\infty}=3$ e $\alpha=0^{\circ}$ è stato calcolato il campo di moto attorno al corpo, i coefficienti di forza e momento, le forze, i momenti, il centro di pressione, il centro aerodinamico e il campo di moto a valle. Infine sono state calcolate le curve di portanza, le polari non viscose e il campo di moto per alcune diverse combinazioni di  $\alpha$ e $M_{\infty}$ allo scopo di evidenziare gli effetti del numero di Mach sull'aerodinamica del profilo.  Il calcolo di tale campo di moto è stato parametrizzato rispetto l'angolo d'attacco del profilo cosicchè lo {\itshape script} sia in grado di valutare per ogni regione la presenza di onde d'urto od onde di espansione in funzione della geometria del profilo alare.

\section{Geometria del Profilo}
La geometria del profilo è stata fornita in forma puntuale mediante i quattro vertici di un quadrilatero
\begin{itemize}
\item Bordo d'attacco A(0,0);
\item Bordo d'uscita B(1,0);
\item C(0.60, 0.02)
\item D(0.35, -0.06)
\end{itemize}

Unendo questi quattro punti si ottiene il profilo in questione come si vede in figura~\vref{figS1}.
\begin{figure}[h!]
\centering
\begin{tikzpicture}
\begin{axis}[ 
xmin=0, 
xmax=1, 
ymin=-0.1,
 ymax=0.1,
 xlabel=$ \frac {x}{c}$, 
ylabel=$ \frac {z}{c}$,
ylabel style={rotate=-90},
ytick={-0.1,-0.05,0,0.05,0.1},
yticklabels={$-0.1$,$-0.05$,$0$,$0.05$,$0.1$},
width=13cm,
 height=2.6 cm,
scale only axis,
grid=major] 
\addplot [black,solid, thick,mark=*,mark options={solid}, mark size=2pt]
file{images/fileDat/Parte_2-Profilo_supersonico/Geometria/Geometria.dat};
\addplot[black,solid, thick,mark=*,mark options={solid}, mark size=2pt]coordinates {
(0,0)
(1,0) };
\node at (axis cs: 0.02, 0.024) {$A$};
\node at (axis cs: 0.6, 0.05) {$C$};
\node at (axis cs: 0.35, -0.085) {$D$};
\node at (axis cs: 0.98, 0.024) {$B$};
%\coordinate (D) at (0.35,-0.06);
%\coordinate (D1) at (0.35,0);
%\coordinate (C) at (0.6,0.02);
%\coordinate (C1) at (0.6,0);
%\draw[thick] (D) -- (D1);
%\draw[thick] (C) -- (C1);
\end{axis}
\end{tikzpicture}
\caption{\footnotesize Punti assegnati e geometria del profilo alare per il volo supersonico}
\label{figS1}
\end{figure}

Tramite lo {\itshape script} MATLAB sono state calcolate tutte le caratteristiche geometriche d'interesse del quadrilatero in analisi le quali, con riferimento alle notazioni presenti in figura~\vref{figS1} sono state riportate nella tabella~\vref{tabS1}.%riferimento tabella

\begin{table} [!h]\centering \rowcolors{1}{}{grigio_chiaro}
\begin{tabular}{c  c  c c }
\toprule
\emph{Lato} & \emph{Lunghezza} [$L/c$] & \emph{Angolo} & \emph{Ampiezza} [deg]  \\
\midrule
$L_1=\overline{AC}$  &  0.600 &   $\theta_1=C\widehat{A}B$ &  $1.91$   \\   
$L_2=\overline{CB}$  &  0.401 &   $\theta_2=C\widehat{B}A$  &  $2.86$ \\ 
$L_4=\overline{AD}$  &  0.355 & $\theta_4= D\widehat{A}B $ &  $9.73 $ \\ 
$L_5=\overline{DB}$  &  0.653 &  $\theta_5= D\widehat{B}A$ &  $5.27 $ \\ 
\bottomrule
\end{tabular}
\caption {\footnotesize Caratteristiche geometriche del profilo}
\label{tabS1}
\end{table}

Nello svolgimento delle analisi riportate nei successivi capitoli sono state ricavate le caratteristiche areodinamiche di profilo riportate in tabella tabella~\vref{tabS1b}.

\begin{table} [!h]\centering \rowcolors{1}{}{grigio_chiaro}
\begin{tabular}{c c}
\toprule
\multicolumn{2}{c}{\emph{Grandezze aerodinamiche}}  \\ 
\midrule
\alphazl &$ -0,7439 ^{\circ}$\\
$\overline{x}_{ac}$ & 0.4350 \\
\bottomrule
\end{tabular}
\caption {\footnotesize Principali grandezze aerodinamiche del profilo per il volo supersonico.}
\label{tabS1b}
\end{table}

%-------------




