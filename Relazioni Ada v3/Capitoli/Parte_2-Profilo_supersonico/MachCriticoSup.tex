% !TeX program = PdfLaTeX
% !TeX root = ../../Elaborati_Aerodinamica_Bruno_Spoti.tex

\chapter{Numero di Mach critico superiore \Msup}
In questo capitolo sono riportati i risultati ottenuti per il calcolo del numero di Mach critico superiore \Msup del profilo per il volo supersonico. Per prima cosa, note le caratteristiche dell'aria, è stato ricostruito l'abbaco d'urto in MATLAB R2016b, per via analitica, riportato in figura~\vref{figS2}.
Successivamente, assegnata la geometria e l'angolo d'attacco ($\alpha$ da $-8^{\circ}$ a $8^{\circ}$, con passo $\alpha=2^{\circ}$), è stato calcolato il numero di Mach critico superiore in quelle condizioni i cui risultati sono riportati in figura~\vref{figS2} e in tabella~\vref{tabS2}.

\begin{figure}[h!]
\centering
\begin{tikzpicture}
\begin{axis}[ 
xmin=0, 
xmax=90, 
ymin=0,
 ymax=50,
 xlabel=$ \varepsilon$, 
ylabel=$ \delta$,
ylabel style={rotate=-90},
width=13cm,
 height=10 cm,
scale only axis,
grid=major] 
\addplot [black,solid]
file{images/fileDat/Parte_2-Profilo_supersonico/MachCriticoSuperiore/AbbacoDurto/delta_vs_epsilon_Mach_1.1616.dat};
\addplot [black,solid]
file{images/fileDat/Parte_2-Profilo_supersonico/MachCriticoSuperiore/AbbacoDurto/delta_vs_epsilon_Mach_1.3636.dat};
\addplot [black,solid]
file{images/fileDat/Parte_2-Profilo_supersonico/MachCriticoSuperiore/AbbacoDurto/delta_vs_epsilon_Mach_1.5657.dat};
\addplot [black,solid]
file{images/fileDat/Parte_2-Profilo_supersonico/MachCriticoSuperiore/AbbacoDurto/delta_vs_epsilon_Mach_1.7677.dat};
\addplot [black,solid]
file{images/fileDat/Parte_2-Profilo_supersonico/MachCriticoSuperiore/AbbacoDurto/delta_vs_epsilon_Mach_1.9697.dat};
\addplot [black,solid]
file{images/fileDat/Parte_2-Profilo_supersonico/MachCriticoSuperiore/AbbacoDurto/delta_vs_epsilon_Mach_2.1717.dat};
\addplot [black,solid]
file{images/fileDat/Parte_2-Profilo_supersonico/MachCriticoSuperiore/AbbacoDurto/delta_vs_epsilon_Mach_2.3737.dat};
\addplot [black,solid]
file{images/fileDat/Parte_2-Profilo_supersonico/MachCriticoSuperiore/AbbacoDurto/delta_vs_epsilon_Mach_2.5758.dat};
\addplot [black,solid]
file{images/fileDat/Parte_2-Profilo_supersonico/MachCriticoSuperiore/AbbacoDurto/delta_vs_epsilon_Mach_2.7778.dat};
\addplot [black,solid]
file{images/fileDat/Parte_2-Profilo_supersonico/MachCriticoSuperiore/AbbacoDurto/delta_vs_epsilon_Mach_2.9798.dat};
\addplot [black,solid]
file{images/fileDat/Parte_2-Profilo_supersonico/MachCriticoSuperiore/AbbacoDurto/delta_vs_epsilon_Mach_3.1818.dat};
\addplot [black,solid]
file{images/fileDat/Parte_2-Profilo_supersonico/MachCriticoSuperiore/AbbacoDurto/delta_vs_epsilon_Mach_3.3838.dat};
\addplot [black,solid]
file{images/fileDat/Parte_2-Profilo_supersonico/MachCriticoSuperiore/AbbacoDurto/delta_vs_epsilon_Mach_3.5859.dat};
\addplot [black,solid]
file{images/fileDat/Parte_2-Profilo_supersonico/MachCriticoSuperiore/AbbacoDurto/delta_vs_epsilon_Mach_3.7879.dat};
\addplot [black,solid]
file{images/fileDat/Parte_2-Profilo_supersonico/MachCriticoSuperiore/AbbacoDurto/delta_vs_epsilon_Mach_3.9899.dat};
\addplot [black,solid]
file{images/fileDat/Parte_2-Profilo_supersonico/MachCriticoSuperiore/AbbacoDurto/delta_vs_epsilon_Mach_4.596.dat};
\addplot [black,solid]
file{images/fileDat/Parte_2-Profilo_supersonico/MachCriticoSuperiore/AbbacoDurto/delta_vs_epsilon_Mach_4.798.dat};
\addplot [black,solid]
file{images/fileDat/Parte_2-Profilo_supersonico/MachCriticoSuperiore/AbbacoDurto/delta_vs_epsilon_Mach_4.1919.dat};
\addplot [black,solid]
file{images/fileDat/Parte_2-Profilo_supersonico/MachCriticoSuperiore/AbbacoDurto/delta_vs_epsilon_Mach_4.3939.dat};
\addplot [black,solid]
file{images/fileDat/Parte_2-Profilo_supersonico/MachCriticoSuperiore/AbbacoDurto/delta_vs_epsilon_Mach_5.dat};
\addplot [black,solid, dashed]
file{images/fileDat/Parte_2-Profilo_supersonico/MachCriticoSuperiore/AbbacoDurto/CurvaMaxAbbaco.dat};
\end{axis}
\end{tikzpicture}
\caption{\footnotesize Abbaco d'urto per l'aria}
\label{figS2}
\end{figure}

\begin{figure}[h!]
\centering
\begin{tikzpicture}
\begin{axis}[ 
xmin=-8, 
xmax=8, 
ymin=1, ymax=2,
 xlabel=$ \alpha$, 
ylabel=\Msup,
ylabel style={rotate=-90},
width=13cm,
 height=7cm,
scale only axis,
grid=major] 
\addplot [black,solid, thick,mark=*,mark options={solid}, mark size=2pt]
file{images/fileDat/Parte_2-Profilo_supersonico/MachCriticoSuperiore/Msup_vs_alpha_Mach.dat};
\end{axis}
\end{tikzpicture}
\caption{\footnotesize Numero di Mach critico superiore al variare di $\alpha$. MATLAB R2016b}
\label{figS3}
\end{figure}

\begin{table} [!h]\centering \rowcolors{1}{}{grigio_chiaro}
\begin{tabular}{c c }
\toprule
$\alpha$ [deg] & \Msup  \\ 
\midrule
-8	&	1.418	\\
-6	&	1.345	\\
-4	&	1.273	\\
-2	&	1.338	\\
0	&	1.411	\\
2	&	1.485	\\
4	&	1.563	\\
6	&	1.645	\\
8	&	1.732	\\
\bottomrule
\end{tabular}
\caption {\footnotesize Numero di Mach critico superiore al variare dell'angolo d'attacco $\alpha$}
\label{tabS2}
\end{table}