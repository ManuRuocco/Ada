\chapter{Carico lungo l'ala: il Metodo di Schrenk}

Il metodo ingegneristico di Schrenk è un metodo semiempirico che consente il calcolo del carico aerodinamico lungo un'ala dritta nell’ipotesi di assenza di fenomeni viscosi a basse velocità.\\
Tale metodo ha il vantaggio di consentire una valutazione piuttosto veloce ed accettabilmente accurata del carico, molto utile in sede di progetto preliminare.\\
\\
Di seguito, in primo luogo, sarà costruito geometricamente il carico lungo la semiala, e in seguito verranno calcolati il carico basico e quello addizionale per poi ottenere il carico totale lungo l’ala. Infine, con i risultati ottenuti, si valuterà il $C_{l(y)}$ lungo l’ala.

\section{Considerazioni preliminari}

Il profilo utilizzato per l’ala rettangolare, a spessore  e corda costante lungo l’ apertura, è un {\bfseries NACA 4412} \cite{prof:jane} con :

\begin {center}
$C_{l_{\alpha}}=0.1170 deg^{-1}=6.704 rad^{-1}$
\end{center}

il $C_{L{\alpha}}$ dell'ala finita sará calcolato con la seguente formula \\

\begin{equation}
\label{eqn:gt}
C_{L_{\alpha}}=\frac {C_{l_{\alpha}}}{{\pi}{\AR}e}
\end{equation}


da cui si ottiene:\\ 

\begin {center}
$C_{L_{\alpha}}=4.746 rad^{-1}=0.08283 deg^{-1}$
\end{center}


Si suppone uno svergolamento di $-3^\circ $ alle estremità, con una variazione lineare di quest'ultimo.\\
Sará utilizzato un asse lungo l'apertura normalizzato rispetto $\frac{b}{2}$, indicato con $\eta$.\\

Il carico aerodinamico totale che agisce sull’ala può essere scomposto nella somma di due contributi:

\begin {itemize}
\item Carico Addizionale: legato alla portanza dell’ala non svergolata,
\item Carico Basico: dipendente esclusivamente dallo svergolamento.
\end{itemize}

\begin{equation}
\label{eqn:carico}
{\gamma}(y)={\gamma}_a(y)+ {\gamma}_b(y)
\end{equation}

Di seguito saranno valutati questi due termini.


\section{Carico Addizionale}

L'ipotesi fondamentale del metodo di Schrenk è quella di valutare il carico addizionale lungo l’apertura come media tra la distribuzione delle corde dell’ala e la distribuzione di corde dell’{\itshape Ala Ellittica Equivalente}, ossia un’ala con forma in pianta ellittica, con stessa apertura e stessa superficie di quella della quale si vuole calcolare il carico.\\
Per graficare questo carico su un piano $c-y$ occorre rappresentare la distribuzione delle corde dell'ala del J450 e quella dell'ala ellittica equivalente, dopo averne calcolato la corda di radice, essendo nulla quella all'estremità. 

\begin{figure} [h]
\centering
\begin{tikzpicture} 
\begin{axis} [ 
xmin=0, 
xmax=5, 
ymin=0,
ymax=2,
xlabel=$y$, 
ylabel=$c$ ,
width=14cm,
height=5.6cm,
scale only axis,
grid=major] 
\addplot [black, thick]
file{immagini/rett.dat};
\addplot [black, thick, dashed, smooth]
file{immagini/ell.dat};
\addplot [black, thick, dashdotdotted,smooth]
file{immagini/ellnew.dat};
\legend {distribuzione corde,ala ellittica equivalente, carico}
\end{axis}
\end{tikzpicture}
\caption{\footnotesize Velivolo Jabiru J450, costruzione del carico addizionale con il metodo Schrenk}
\end{figure}

\noindent \\

Per un ampio intervallo di assetti, il carico addizionale risulta proporzionale al $C_L$, per cui si ha 

\begin{equation}
\label{eqn:one}
\gamma_a(y)=C_L \gamma_{a1}(y)
\end{equation}

Dalle \ref{eqn:one}, \ref{eqn:two}, \ref{eqn:tree}


\begin{equation}
\label{eqn:two}
\gamma_{a1}(\eta)=\frac{cC_{l_a}}{2b}=\frac{c+c_{ell}}{4b}
\end{equation}

\noindent \\

\begin{equation}
\label{eqn:tree}
cC_{la1}=\frac{c(y)+c_{ell}(y)}{2}
\end{equation}

Si ricava la distribuzione del carico addizionale riportata nel grafico che segue


\begin{figure} [h]
\centering
\begin{tikzpicture} 
\begin{axis} [ 
xmin=0, 
xmax=1, 
ymin=0,
ymax=0.03,
xlabel=$\eta$, 
ylabel=$\gamma_a$ ,
width=14cm,
height=7cm,
scale only axis,
grid=major] 
\addplot [black,very thick,smooth]
file{immagini/gammaadd.dat};
\end{axis}
\end{tikzpicture}
\caption{\footnotesize Velivolo Jabiru J450, carico addizionale lungo la semiala per {\bfseries $C_L=0.3$}}
\end{figure}


\section{Carico Basico}

Il metodo di Schrenk assume che il carico basico dovuto allo svergolamento si dimezza rispetto allo svergolamento realmente esistente.

\begin{equation}
\label{eqn:bas}
cC_{l_b}=cC_{l_{\alpha}} \frac{1}{2}(\varepsilon_y-[\alpha]_{C_L=0})
\end{equation}

\noindent \\

Al fine di poter calcolare il carico basico, si suppone uno svergolamento di $-3^\circ $ alle estremità e svergolamento nullo alla radice, con una legge di svergolamento lineare. L’angolo di portanza nulla dell’ala risulta essere $ \alpha_{ZL}=-1.5^\circ$


\begin{figure} [H]
\centering
\begin{tikzpicture} 
\begin{axis} [ 
xmin=0, 
xmax=1, 
ymin=-0.01,
ymax=0.01,
xlabel=$\eta$, 
ylabel=$\gamma_b$ ,
width=14cm,
height=7cm,
scale only axis,
grid=major] 
\addplot [black, thick]
file{immagini/caricobas.dat};
\end{axis}
\end{tikzpicture}
\caption{\footnotesize Velivolo Jabiru J450, carico basico lungo la semiala, $\varepsilon_r=0^\circ$,$\varepsilon_t=-3^\circ$ }
\end{figure}



\section{Carico Totale}

Nell'ipotesi di svergolamento, il carico totale agente sull'ala sará la somma del carico basico e del carico addizionale.




\begin{figure} [h]
\centering
\begin{tikzpicture} 
\begin{axis} [ 
xmin=-1, 
xmax=1, 
ymin=0,
ymax=0.03,
xlabel=$\eta$, 
ylabel=$\gamma$ ,
width=14cm,
height=7.5cm,
scale only axis,
grid=major] 
\addplot [black,very thick,smooth]
file{immagini/caricotot.dat};
\end{axis}
\end{tikzpicture}
\caption{\footnotesize Velivolo Jabiru J450, carico totale lungo l'ala per {\bfseries $C_L=0.3$}}
\end{figure}

\section{Coefficiente di portanza lungo l'apertura alare}

Noto il carico sull'ala è  possibile calcolare la distribuzione del coefficiente di portanza lungo l'ala dalla seguente formula\\

\begin{equation}
\label{eqn:clala}
C_l(y)=\frac{2b\gamma(y)}{c(y)}
\end{equation}

\begin{figure} [h]
\centering
\begin{tikzpicture} 
\begin{axis} [ 
xmin=-1, 
xmax=1, 
ymin=0,
ymax=1.3,
xlabel=$\eta$, 
ylabel=$C_l$ ,
width=13cm,
height=6.3cm,
scale only axis,
grid=major] 
\addplot [black, thin,smooth]
file{immagini/cladd03.dat};
\addplot [black, very thick,smooth]
file{immagini/cladd1.dat};
\legend {$C_L=0.3$,$C_L=1$}
\end{axis}
\end{tikzpicture}
\caption{\footnotesize Velivolo Jabiru J450, andamento del coefficiente di portanza lungo l'apertura alare, ala non svergolata}
\end{figure}
\begin{figure} [h]
\centering
\begin{tikzpicture} 
\begin{axis} [ 
xmin=-1, 
xmax=1, 
ymin=0,
ymax=1.3,
xlabel=$\eta$, 
ylabel=$C_l$ ,
width=13cm,
height=6.3cm,
scale only axis,
grid=major] 
\addplot [black, thin,smooth]
file{immagini/cl1schrenk.dat};
\addplot [black, very thick,smooth]
file{immagini/clUNOschrenk.dat};
\legend {$C_L=0.3$,$C_L=1$}
\end{axis}
\end{tikzpicture}
\caption{\footnotesize Velivolo Jabiru J450, andamento del coefficiente di portanza lungo l'apertura alare, ala con svergolamento}
\end{figure}



