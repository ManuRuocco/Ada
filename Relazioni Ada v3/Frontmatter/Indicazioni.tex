% !TeX program = PdfLaTeX
% !TeX root = ../Elaborato.tex
\begin{center}
\bfseries \normalsize Indicazioni per lo sviluppo delle esercitazioni a casa
\end{center}

Il rispetto di queste indicazioni è tassativo. In presenza di difformità non prenderò in considerazione le relazioni.

Ogni cosa riportata va letta con molta attenzione prima di essere sottoposta alla mia attenzione: non conviene ``usare'' un docente come correttore di bozze.

\textbf{STESURA DEL TESTO (CON O SENZA WORD PROCESSOR).} È richiesta un'esposizione strutturata piuttosto che narrativa.
Pertanto descrivere sinteticamente ed in sequenza \\[-1.6em]
\begin{itemize}
\setlength\itemsep{0em}
\item lo scopo
\item lo sviluppo
\item l'applicazione
\item le conclusioni
\end{itemize}
indicando gli strumenti (tecnici, informatici o scientifici) utilizzati per lo sviluppo e la stesura.

È vietato riprodurre, anche in parte, la teoria alla base dell'esercizio: limitarsi all'indicazione bibliografica.

La lunghezza, in facciate, del corpo del resoconto del lavoro a casa (escludendo quindi titolo, indice e lista dei simboli) va
contenuta al massimo.

\textbf{INDICAZIONI PARTICOLARI.} Il fascicolo che contiene gli esercizi deve essere curato, preciso, elegante, e pertanto \\[-1.6em]
\begin{itemize}
\setlength\itemsep{0em}
\item i risultati numerici vanno riportati con la giusta accuratezza: porre ESTREMA attenzione all'aspetto delle cifre significative
\item ogni rappresentazione grafica deve essere pertinente
\item riportare sempre il sommario dei risultati in quadri sinottici od in opportuni grafici
\item FIGURE/DIAGRAMMI. Figure in bianco, nero e toni di grigio (immagini e foto riprese da sorgenti bibliografiche, compresa la rete, potranno essere a colori). Inserire nel testo oppure alla fine, numerando e spaziando per bene, nel rispetto e con indicazione delle scale, con una legenda esauriente (=con tutte le indicazioni), senza sovrapporre la legenda ai grafici, usare simboli adeguatamente grandi. Il formato deve essere umano e l'assetto verticale. Ogni risultato in figura va commentato (nel testo od anche in didascalia). Il Cd/CD va misurato in Drag Count e parte sempre da zero (lo stesso vale per la resistenza), ingrandire le polari nelle regioni di bassa resistenza
\item Il disegno del profilo: LE SCALE (!), produrre una figura della larghezza utile della pagina, il tratto deve essere ``corretto''
\item evitare per quanto possibile termini in lingua diversa dall'italiano (un termine irrinunciabile di altra lingua va scritto in corsivo), evitare tout court versioni italianizzate di termini di altre lingue
\item nella stesura informatica lasciare un spazio bianco dopo i caratteri .,;?!; in stampa lasciare 3.5 cm a sx, 2 cm a dx
\item eventuali formule vanno numerate
\item non è necessario (ma può essere utile) riportare la lista dei simboli
\item impiegare sempre una terminologia appropriata
\item stare attenti ad evitare il costrutto “: (due punti) seguito da una figura o da una tabella”
\item CFD. Le scale in toni di grigio. Congruità dei confronti con Xfoil: parità di Cl, rispetto dei limiti di validità.
\item Scrivere sempre “numero di” Mach/Reynolds e non “Mach/Reynolds”
\end{itemize}

\textbf{PRESENTAZIONE.} Esercizi ed elaborati vanno presentati in un fascicolo non rilegato, indicando in copertina cognome, nome e matricola, insieme all'elenco di tutti gli esercizi in sviluppo o già convalidati, e riportando in seconda pagina le \textbf{INDICAZIONI PER LO SVILUPPO DELLE ESERCITAZIONI A CASA.} La forma è da me valutata in modo paritetico rispetto ai contenuti (e dunque leggere ogni cosa con molta attenzione prima di sottopormela).

\textbf{CONTROLLO E CORREZIONE.} Io controllo e correggo solo testi -parziali o completi- purché già scritti in una forma definitiva (i.e., non in bozza). Ovviamente il proponente procederà ad una preliminare autoverifica anche (e sopratutto) per gli aspetti formali... Interromperò il controllo di un esercizio alla prima violazione di una delle regole sopra riportate. È possibile sottopormi via mail il testo da controllare (in formato .pdf, dimensione <500kb). 